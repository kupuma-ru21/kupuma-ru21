% !TEX program = xelatex
\documentclass{article}
\usepackage{hyperref}
\usepackage{fontspec}
\usepackage{xeCJK}
\usepackage{enumitem}
\usepackage[margin=1in]{geometry}
\setCJKmainfont[
  BoldFont = IPAGothic
]{IPAexGothic}
\setmainfont[
  BoldFont = IPAexGothic
]{IPAexGothic}
\setCJKmonofont{IPAGothic}
\linespread{0.2}
\xeCJKsetup{
  CJKglue = {\hskip -0.05em}
}
\setlist[itemize]{itemsep=0.3em, topsep=0.3em, parsep=0em}

\XeTeXinterchartokenstate=1
\newXeTeXintercharclass\english
\XeTeXcharclass`A=\english
\XeTeXcharclass`B=\english
\XeTeXcharclass`C=\english
\XeTeXcharclass`D=\english
\XeTeXcharclass`E=\english
\XeTeXcharclass`F=\english
\XeTeXcharclass`G=\english
\XeTeXcharclass`H=\english
\XeTeXcharclass`I=\english
\XeTeXcharclass`J=\english
\XeTeXcharclass`K=\english
\XeTeXcharclass`L=\english
\XeTeXcharclass`M=\english
\XeTeXcharclass`N=\english
\XeTeXcharclass`O=\english
\XeTeXcharclass`P=\english
\XeTeXcharclass`Q=\english
\XeTeXcharclass`R=\english
\XeTeXcharclass`S=\english
\XeTeXcharclass`T=\english
\XeTeXcharclass`U=\english
\XeTeXcharclass`V=\english
\XeTeXcharclass`W=\english
\XeTeXcharclass`X=\english
\XeTeXcharclass`Y=\english
\XeTeXcharclass`Z=\english
\XeTeXcharclass`a=\english
\XeTeXcharclass`b=\english
\XeTeXcharclass`c=\english
\XeTeXcharclass`d=\english
\XeTeXcharclass`e=\english
\XeTeXcharclass`f=\english
\XeTeXcharclass`g=\english
\XeTeXcharclass`h=\english
\XeTeXcharclass`i=\english
\XeTeXcharclass`j=\english
\XeTeXcharclass`k=\english
\XeTeXcharclass`l=\english
\XeTeXcharclass`m=\english
\XeTeXcharclass`n=\english
\XeTeXcharclass`o=\english
\XeTeXcharclass`p=\english
\XeTeXcharclass`q=\english
\XeTeXcharclass`r=\english
\XeTeXcharclass`s=\english
\XeTeXcharclass`t=\english
\XeTeXcharclass`u=\english
\XeTeXcharclass`v=\english
\XeTeXcharclass`w=\english
\XeTeXcharclass`x=\english
\XeTeXcharclass`y=\english
\XeTeXcharclass`z=\english

\begin{document}

\noindent
\textbf{\large 名前: 木村 浩一(きむら こういち)} \textbar\
\href{https://github.com/kupuma-ru21}{GitHub} \textbar\
\href{https://kupuma-ru21.com/}{Portfolio} (\href{https://github.com/kupuma-ru21/portfolio}{Repository}) \textbar\
\href{https://github.com/kupuma-ru21/kupuma-ru21/blob/main/OSS.md}{OSS} \textbar\
\href{https://github.com/kupuma-ru21/kupuma-ru21/blob/main/TECH_BLOG.md}{Tech Blog}

\vspace{-1em}

\section*{職務経歴}

\noindent
\textbf{株式会社Sorajima | フルスタックエンジニア | 2024\kern-0.1em年5\kern-0.1em月 - 現在 | 東京, 日本(リモート)}

\vspace{0.15em}

\textbf{\href{https://sorajimatoon.com/}{ソラジマTOON}} / ソラジマTOONの管理画面のフロント・バックエンド開発

\begin{itemize}[leftmargin=1.5em]

  \item Lighthouseのパフォーマンススコアを35\% → 75\%に改善\\
  リクエスト数削減 | 画像読み込み最適化(\texttt{<img>}: \texttt{fetchPriority}, \texttt{loading}使用) | \texttt{Promise.all}で非 同期処理の並列化 | レイアウトシフト解消 | Componentを\texttt{Controlled}⇒\texttt{Uncontrolled}に変更しレンダリングを削除


  \item 楽観的UIを使用しUXの改善

  \item DX改善:Componentを\href{https://kentcdodds.com/blog/compound-components-with-react-hooks}{CompoundComponent}で再設計し、柔軟にUIを表現できるよう変更

  \item APIの追加、改修、バグ修正を行い、バックエンドエンジニアとの会話コスト削減に貢献

  \item プロジェクト新規立ち上げを担当: 未経験だったが興味があり、個人でReact・Go・GCPで\href{https://github.com/kupuma-ru21/portfolio}{Webアプリ}を開発、その後本番プロジェクトを円滑に進行

\end{itemize}

\noindent
\makebox[3em][l]{\textbf{技術:}}%
\parbox[t]{\dimexpr\linewidth-2em\relax}{
React.js, Next.js, Remix, Chakra-UI, Storybook, GraphQL, GraphQL-Codegen, Apollo Client, URQL, React-Hook-Form, Zod, ESLint, Prettier, PNPM, Git, Slack, Jira, Figma, Golang, Ent, Gqlgen
}

\vspace{0.15em}
\noindent\rule{\linewidth}{0.4pt}
\vspace{0.15em}

\noindent
\textbf{Buysell Technologies | フルスタックエンジニア | 2023\kern-0.1em年2\kern-0.1em月 - 2023\kern-0.1em年12\kern-0.1em月 | 東京, 日本(リモート)}

\vspace{0.15em}

\textbf{買取店舗で使用する業務システムのフロント・バックエンド開発}

\begin{itemize}[leftmargin=1.5em]

  \item 日付ライブラリの再選定とリプレイス: day.jsの\href{https://github.com/iamkun/dayjs/issues/1236}{パフォーマンス問題}により、date-fnsへ移行\\
  date-fnsの理由は\href{https://npmtrends.com/date-fns-vs-dayjs-vs-luxon}{install数}が多く情報が豊富、型安全なTypeScript製、\href{https://github.com/date-fns/date-fns/graphs/contributors}{メンテナンスが活発}な点

  \item DX改善:\href{https://speakerdeck.com/sonatard/coheision-coupling?slide=29}{論理的凝集}が起きていたComponentを責務ごとに分離し、可読性と保守性を向上

  \item APIの新規追加・ER図の更新:
  未経験ながらバックエンドに挑戦。SQLBoilerを主に学習、APIとテストを期日内に実装。既存テーブルにカラム追加、ER図も更新

\end{itemize}

\noindent
\makebox[3em][l]{\textbf{技術:}}%
\parbox[t]{\dimexpr\linewidth-2em\relax}{
React.js, Next.js, Storybook, GraphQL, GraphQL-Codegen, Apollo Client, Material-UI, React-Hook-Form, Zod, Vitest, ESLint, Prettier, PNPM, Git, Slack, Jira, Figma, Golang, SQLBoiler, Gqlgen
}


\end{document}
